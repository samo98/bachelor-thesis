\chapter{Problém konsenzu}

Problém konsenzu vyžaduje dohodu medzi viacerými entitami na jednej a tej
istej \textit{hodnote} (\textit{hodnota} môže byť ľubovoľný typ informácie).
Tento problém začne byť zaujímavý keď pripustíme, že niektoré z entít
podieľajúcich sa na konsenze budú nespoľahlivé alebo dokonca zlyhajú. V takomto
prípade budeme očakávať od riešenia (súbor pravidiel, ktoré keď budú uzly dodržiavať
dosiahnu konsenzus, ďalej ho budeme nazývať \textit{protokol}),
že entity, ktoré ešte pracujú podľa protokolu budú schopné dosiahnuť konsenzus
aj napriek týmto zlyhaniam.

Aby sme vylúčili triviálne riešenia (všetky entity odsúhlasia konsenzus na vopred 
dohodnutej konštante a podobne), výsledná dohodnutá hodnota musí závisieť na vstupe
daných entít, preto pred začatím procesu dohadovania konsenzu
musia entity navrhnúť kandidátov. \textit{Kandidáti} budú hodnoty
z ktorých entity následne vyberú jedného, na ktorom sa všetci zhodnú a ustanovia
konsenzus.
Celý konsenzus protokol musí mať teda tri fázy: entity navrhnú kandidátov na hodnoty,
entity spolu komunikujú a na konci každá entita oznámi, na ktorom kandidátovi
na hodnotu sa dohodli, pričom všetky nezlyhané entity oznámia toho istého kandidáta.

Tento problém je jeden z najzákladnejších problémov informatiky.
Mnohí informatici ho ho riešili už niekoľko desaťročí a dokázali prísť
na mnohé zaujímavé dôsledky (napríklad \textit{dôkaz nemožnosti konsenzu}
\cite{fischer1982impossibility}). Takto zadaný problém je však veľmi všeobecný
a dá sa riešiť na rôznych modeloch, kde si konkretizujeme naše entity, spôsob
komunikácie medzi nimi a ďalšie vlastnosti siete entít. A teda napriek
\textit{dôkazu nemožnosti konsenzu} \cite{fischer1982impossibility} dokážeme dnes
za istých podmienok spravovať distribuované systémy.

\section {Model siete v našej práci}

Teraz si zadefinujeme model siete, ktorý budeme uvažovať v celej práci.
Napriek tomu, že náš model bude teoretický, všetky rozhodnutia konkretizovania
modelu sa budeme snažiť robiť na základe skúseností z praxe aby prípadná
implementácia do reálneho distribuovaného systému nebola obtiažna, ale zároveň
nebudeme brať mnohé reálne problémy do modelu, ktoré priamo nesúvisia s
problémom konsenzu a iba by komplikovali konsenzus a dôkazy správnosti v ňom.

V našej práci nás bude tento problém zaujímať hlavne v distribuovaných systémoch.
A preto v našom modeli entity budú jednotlivé počítače v sieti.
V celej práci budeme počítače (alebo ľubovoľné entity) v sieti nazývať 
\textit{uzly}.
\\*
\textit{Sieť} je množina uzlov, ktoré sú poprepájané každý s každým a vedia si
teda medzi sebou vymieňať ľubovoľné správy. Nebudeme dávať žiadne obmedzenia
na veľkosť správ a ani ich podobu. Dokonca nebudeme mať žiadne predpoklady na
poradie doručenia správ. Avšak budeme predpokladať, že správy budú
v konečnom čase doručené v nezmenenej podobe (takýto predpoklad nám v reálnej
sieti vie ponúknuť kvalitné šifrovanie).
Celá komunikácia bude prebiehať asynchrónne a teda nepredpokladáme žiadnu
synchronizáciu uzlov.

Takisto v našom modeli povolíme byzantínske chyby uzlov.
Čo znamená, že na správanie uzlov nebudeme dávať žiadne predpoklady.
Toto zodpovedá aj reálnemu svetu, kde uzol môže zlyhať po hardvérovej alebo
softvérovej chybe alebo dokonca nepriateľ vie upraviť vykonávaný kód (na tento
upravený kód už nemôžeme mať žiadne predpoklady) na uzle.
V poslednom z prípadov keď nepriateľ upraví vykonávaný kod na uzle budeme
hovoriť, že útočník sa \uv{zmocnil} uzlu.

A teda uzly budeme vždy triediť do dvoch základných kategórií, tie čo sa správajú
podľa dohodnutého protokolu (budeme hovoriť, že sa držia protokolu) a tie ktoré
porušujú protokol (nebudeme rozlišovať z akého dôvodu). Uzlu ktorý porušuje protokol budeme hovoriť, že \textit{zlyhal}.

\section {Od teoretického modelu k bankovému sektoru}

\label{kap:theorytobank}

Na prvý pohľad nečakane, môžeme konsenzus využiť vo finančnej sieti.
Schvaľovanie konsenzu zodpovedá schvaľovaniu finančných transakcií a uzly sú
jednotliví používatelia peňazí. Stav siete je vlastne zoznam uskutočnených
transakcií (konkrétne implementácie ukladania a manipulovania so stavom necháme
na jednotlivých protokoloch).
Tento zoznam sa zvykne nazývať \textit{bloková reťaz}, keďže každú
transakciu vieme zapísať ako jeden blok dát o nej a tieto transakcie vieme
zoradiť za sebou do reťaze, ktorú postupne rozširujeme a na jej koniec
zapisujeme nové transakcie.
Dnešná celosvetová finančná sieť v takomto jednoduchom ponímaní neexistuje. Na
svete existuje veľa rôznych mien a prevody peňazí medzi menami nie sú až také
jednoduché ako prevody v jednej mene.

Celkovo svet vyzerá ako mnoho malých (alebo väčších) sietí s centrálnou autoritou,
ktorou sú banky, ktoré nám naše transakcie musia schváliť.
A teda prevody peňazí z jedného konca sveta na druhý sú náročné.
Neskôr si popíšeme aké sú nevýhody tohto systému s jednou autoritou (nazývané aj
centralizované systémy) a aké sú možnosti náhrady tohto systému.

Na záver podkapitoly si zhrňme všetky očakávania od finančnej siete, ktoré
aktuálna finančná sieť nespĺňa a v ďalšej podkapitole si predstavíme protokoly
na nájdenie konsenzu, ktoré sa snažia aspoň niektoré vlastnosti spĺňať (a naopak
tým zlým sa vyhýbať).
V kapitole \ref{kap:stellar} potom predstavíme \textit{Stellar konsenzus protokol},
ktorý bude všetky tieto vlastnosti spĺňať a budeme sa mu v práci venovať hlbšie.

\begin{itemize}
\item  \textbf{Jedna globálna} -- chceme nahradiť veľa aktuálnych sietí jednou
celosvetovo použiteľnou
\item  \textbf{Decentralizovaná} -- nechceme nechať štát a banky kontrolovať náš
finančný tok, chceme nech všetci zapojený v sieti dohliadajú na dodržovanie
pravidiel
\item  \textbf{Nízka latencia} -- nech už posielame peniaze kdekoľvek chceme, aby
bola transakcia vykonaná v čo najkratšom čase
\item  \textbf{Otvorené členstvo v sieti} -- žiadne prekážky k vytvoreniu si
účtu, tak ako dnes v mnohých bankách, možnosť otvoriť si účet kdekoľvek na svete
v čo najkratšom čase
\item  \textbf{Otvorené členstvo medzi schvaľovateľmi} -- žiadne prekážky ku
zapojeniu sa do schvaľovaciemu procesu
\item  \textbf{Flexibilná dôvera} -- najradšej by sme si sami určili, komu
dôverujeme a neboli zaviazaný bankovým inštitúciám, ktoré môžu rôznymi
spôsobmi ovplyvňovať finančnú sieť
\item  \textbf{Odolnosť voči zlyhaniam} -- sieť by nemala prestať fungovať aj
napriek zlyhaniam niektorých uzlov v sieti, sieť je odolnejšia keď aj zlyhanie
viacerých uzlov neovplyvní jej chod
\end{itemize}

\pagebreak

\section {Centralizovaný protokol a jeho problémy}

Najčastejším a najjednoduchším riešením tohto problému je \textbf{centralizácia}.
Systém navrhneme tak, že obsahuje jeden hlavný uzol, označme ho $\textbf{A}$.
Tento uzol je autorita a udržiava stav celého systému.
Teraz pokiaľ chce iný uzol zmeniť stav systému alebo uložiť dáta musí
nadviazať spojenie s uzlom $\textbf{A}$ a požiadať o zmenu stavu. Uzol $\textbf{A}$ túto žiadosť
zvaliduje (aby nebola porušená konzistentnosť, alebo ináč narušený chod systému)
a tento stav uloží. Ostatné uzly už budú potom informované o novom stave pri
akejkoľvek komunikácií s uzlom $\textbf{A}$.
Uzol $\textbf{A}$ je teda zárukou toho, že údaje v systéme sú korektné a systém vie stále
vykonávať svoju úlohu.
Nech sa teda útočník zmocní ľubovoľného uzla okrem $\textbf{A}$, alebo nech ľubovoľný uzol
zlyhá, systém môže stále pokračovať v svojej úlohe.
Obrovská nevýhoda však je, že práve uzol $\textbf{A}$ je slabina celého systému. Keď zlyhá
uzol $\textbf{A}$ celý systém nie je schopný pokračovať. Ďalšou veľkou nevýhodou je, že
všetky požiadavky systému musí validovať práve uzol $\textbf{A}$ a teda môže dôjsť k jeho
preťaženiu.

\section {Známe decentralizované protokoly}

\begin{flushleft}\textbf {Dôkaz prácou}\end{flushleft}
\vspace{-4mm}
Jedným z prvých pokusov o zlepšenie finančného systému je Bitcoin
\cite{nakamoto2009bitcoin}.
Bol to krok dobrým smerom, keďže ako prvá kryptomena dokázala fungovať na 
decentralizovanej myšlienke a teda sa vyhla používaniu jednej centrálnej autority.
\\*
Napriek tomu nedokázal vyriešiť všetky problémy popísané
vyššie.
\\*
Je založený na myšlienke dôkazu prácou \cite{dwork1992pricing}.
Veľkou nevýhodou je jeho neekologickosť, keďže plytvá zdrojmi. Existuje odhad z
roku 2014, kedy Bitcoin skonzumoval toľko elektrickej energie ako celá krajina
Írsko \cite{malone2014bitcoin}. Totiž na schválenie transakcie musí schvaľovateľ vyriešiť ťažkú
matematickú šifru, čo tiež má za následok pomalosť transakcií. Takisto nie každý
sa môže stať schvaľovateľom, keďže schvaľovateľ musí disponovať veľkou
výpočtovou silou aby dokázal vyriešiť matematické šifry.

\begin{flushleft}\textbf {Dôkaz bohatstvom}\end{flushleft}
\vspace{-4mm}
Jeho alternatívou je napríklad protokol dôkazu bohatstvom \cite{king2012ppcoin}. Tento
protokol sa vyhýba problému protokolu dôkazu prácou, ktorého transakcie boli
pomalé.
\\*
Jeden z najväčších problémov je takzvaný \mbox{\uv{nič v stávke}\thinspace
problém (angl. nothing-at-stake).}
Keďže schvaľovanie transakcií je lacné (časovo aj výpočtovo),
nič nebráni schvaľovateľom podporovať rôzne blokové reťaze (transakčné
histórie).

\newpage

\begin{flushleft}\textbf {Praktická byzantínska tolerancia voči chybám}\end{flushleft}
\vspace{-4mm}
Ďalším diametrálne odlišným prístupom je protokol \textit{praktickej byzantínskej
tolerancie voči chybám} \cite{castro1999practical}.
Tento protokol na rozdiel od vyššie spomínaných nemá žiadne predpoklady na
správanie sa nepriateľských uzlov (toto správanie označujeme ako byzantínske a uzly,
ktoré sa tak správajú označujeme ako Byzantíncov).
Takisto dosahuje zaručene nízku latenciu.
\\*
Nespĺňa však našu požiadavku o otvorenom členstve. Sieť sa spolieha na to, že
útočník nemôže rýchlo a jednoducho pridať veľa uzlov do siete a tak sa
zmocniť dostatočne veľkého množstva uzlov, čím by vedel ovplyvňovať hodnoty na ktorých
sieť dosiahne konsenzus.
Takýmto útokom sa tiež zvykne hovoriť \textit{Sybil útoky}.
A preto zaradenie uzla do siete musí prejsť schvaľovacím procesom.
Vo všeobecnosti, siete používajúce tento protokol, nechávajú schvaľovanie
členstva na centrálnej autorite.
