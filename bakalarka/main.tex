\documentclass[12pt, twoside]{book}
\usepackage[a4paper,top=2.5cm,bottom=2.5cm,left=3.5cm,right=2cm]{geometry}
\usepackage[utf8]{inputenc}
\usepackage[T1]{fontenc}
\usepackage{graphicx}
\usepackage{url}
\usepackage[hidelinks,breaklinks]{hyperref}
\usepackage[dvipsnames]{xcolor}
\usepackage{mathtools}
\usepackage{amssymb}
\usepackage[slovak]{babel}
\usepackage{lmodern}
\usepackage{algorithm}
\usepackage[noend]{algpseudocode}
\linespread{1.25}

\makeatletter
\renewcommand{\ALG@name}{Algoritmus}
\makeatother

% -------------------
% --- Definicia zakladnych pojmov
% --- Vyplnte podla vasho zadania
% -------------------
\def\mfrok{2019}
\def\mfnazov{Analýza odolnosti Stellar konsenzus protokolu voči Byzantínskym chybám}
\def\mftyp{Bakalárska práca}
\def\mfautor{Samuel Sládek}
\def\mfskolitel{RNDr. Tomáš Kulich, PhD.}

\def\mfmiesto{Bratislava, \mfrok}

%aj cislo odboru je povinne a je podla studijneho odboru autora prace
\def\mfodbor{2508 Informatika} 
\def\program{ Informatika }
\def\mfpracovisko{ Katedra informatiky }

\begin{document}     
\frontmatter


% -------------------
% --- Obalka ------
% -------------------
\thispagestyle{empty}

\begin{center}
\sc\large
Univerzita Komenského v Bratislave\\
Fakulta matematiky, fyziky a informatiky

\vfill

{\LARGE\mfnazov}\\
\mftyp
\end{center}

\vfill

{\sc\large 
\noindent \mfrok\\
\mfautor
}

\eject % EOP i
% --- koniec obalky ----

% -------------------
% --- Titulný list
% -------------------

\thispagestyle{empty}
\noindent

\begin{center}
\sc  
\large
Univerzita Komenského v Bratislave\\
Fakulta matematiky, fyziky a informatiky

\vfill

{\LARGE\mfnazov}\\
\mftyp
\end{center}

\vfill

\noindent
\begin{tabular}{ll}
Študijný program: & \program \\
Študijný odbor: & \mfodbor \\
Školiace pracovisko: & \mfpracovisko \\
Školiteľ: & \mfskolitel \\
% Konzultant: & \mfkonzultant \\
\end{tabular}

\vfill


\noindent \mfmiesto\\
\mfautor

\eject % EOP i


% --- Koniec titulnej strany


% -------------------
% --- Zadanie z AIS
% -------------------
% v tlačenej verzii s podpismi zainteresovaných osôb.
% v elektronickej verzii sa zverejňuje zadanie bez podpisov

\newpage
\thispagestyle{empty}
\hspace{-2cm}\includegraphics[width=1.1\textwidth]{images/zadanie}

% --- Koniec zadania

\frontmatter

% -------------------
%   Poďakovanie - nepovinné
% -------------------
\setcounter{page}{3}
\newpage

\begin{flushleft}\textbf{\Large Poďakovanie}\end{flushleft}
Chcel by som sa poďakovať môjmu školiteľovi RNDr. Tomášovi Kulichovi, PhD.
za návrh témy a podpory v objavovaní pre mňa nového sveta kryptomien.

Rovnako Patrikovi Bakovi za skvelé rady, postrehy a najmä možnosti
vyrozprávania sa o nástrahách čakajúcich v priebehu vyhotovovania práce.

% --- Koniec poďakovania

% -------------------
%   Abstrakt - Slovensky
% -------------------
\newpage 
\section*{Abstrakt}

Kryptomena Stellar pred pár rokmi prešla na nový protokol, ktorý má vylepšovať
doteraz známe konsenzus protokoly najmä zjednodušením možnosti vstupu do siete
pre novú entitu.
Táto práca analyzuje odolnosť tohto protokolu voči zlyhaniam jednotlivých
entít spolupracujúcich na konsenze. Najprv ukáže, že pre danú sieť nevieme
v polynomiálnom čase povedať pri koľkých zlyhaniach bude sieť stále bezpečná
a ani pri koľkých zlyhaniach bude sieť stále schopná dohodnúť sa na konsenze.
Neskôr sa pozrie na niektoré zaujímavé typy sietí pri ktorých odolnosť vyjadriť
vieme.
Vytvorí algoritmus na presné vyjadrenie odolnosti, respektíve rýchlejší
algoritmus na odhady zhora pre odolnosť siete.
Na záver rozanalyzuje bezpečnosť a odolnosť náhodne vytvorených sietí.
Túto analýzu ukončí odporúčaním volenia parametrov siete aby bola sieť schopná
prežiť čo najviac zlyhaní.

\paragraph*{Kľúčové slová:} Stellar konsenzus protokol, byzantínske chyby,
kvórum, odolnosť siete
% --- Koniec Abstrakt - Slovensky


% -------------------
% --- Abstrakt - Anglicky 
% -------------------
\newpage 
\section*{Abstract}

A few years ago, cryptocurrency Stellar has switched to a new protocol
to improve the consensus protocols known so far. In particular by simplifying
network entry for a new entity.
This thesis analyzes the resilence of this protocol to Byzantine failures.
Firstly it shows impossibility to evaluate number of failures the network is
still safe or the number of failures, the network can still agree on consensus in
polynomial time.
Later, it considers some interesting types of networks in which we are able
to calculate resilence.
It creates algorithm to accurately express resilence, or even faster method
for estimating network resilence quickly.
At the end it analyzes security and resilence of randomly created networks.
This analysis ends with recommendations for choosing network parameters to
make resilence as high as possible.



\paragraph*{Keywords:}  Stellat consensus protocol, Byzantine failures,
quorum, network resilence

% --- Koniec Abstrakt - Anglicky

% -------------------
% --- Predhovor - v informatike sa zvacsa nepouziva
% -------------------
%\newpage 
%\thispagestyle{empty}
%
%\huge{Predhovor}
%\normalsize
%\newline
%Predhovor je všeobecná informácia o práci, obsahuje hlavnú charakteristiku práce 
%a okolnosti jej vzniku. Autor zdôvodní výber témy, stručne informuje o cieľoch 
%a význame práce, spomenie domáci a zahraničný kontext, komu je práca určená, 
%použité metódy, stav poznania; autor stručne charakterizuje svoj prístup a svoje 
%hľadisko. 
%
% --- Koniec Predhovor


% -------------------
% --- Obsah
% -------------------

\newpage 

\tableofcontents

% ---  Koniec Obsahu

% -------------------
% --- Zoznamy tabuliek, obrázkov - nepovinne
% -------------------

\newpage 

\listoffigures
%\listoftables

% ---  Koniec Zoznamov

\mainmatter


\input uvod.tex 

\input consenzus.tex

\input stellar.tex

\input security.tex

\input particular_networks.tex

\input random.tex

\input zaver.tex

% -------------------
% --- Bibliografia
% -------------------


\newpage	

\backmatter

\thispagestyle{empty}
\nocite{*}
\clearpage

\addcontentsline{toc}{chapter}{Literatúra} % rucne pridanie do obsahu
\bibliographystyle{plain}
\bibliography{literatura}

%Prípadne môžete napísať literatúru priamo tu
%\begin{thebibliography}{5}
 
%\bibitem{br1} MOLINA H. G. - ULLMAN J. D. - WIDOM J., 2002, Database Systems, Upper Saddle River : Prentice-Hall, 2002, 1119 s., Pearson International edition, 0-13-098043-9

%\bibitem{br2} MOLINA H. G. - ULLMAN J. D. - WIDOM J., 2000 , Databasse System implementation, New Jersey : Prentice-Hall, 2000, 653s., ???

%\bibitem{br3} ULLMAN J. D. - WIDOM J., 1997, A First Course in Database Systems, New Jersey : Prentice-Hall, 1997, 470s., 

%\bibitem{br4} PREFUSE, 2007, The Prefuse visualization toolkit,  [online] Dostupné na internete: <http://prefuse.org/>

%\bibitem{br5} PREFUSE Forum, Sourceforge - Prefuse Forum,  [online] Dostupné na internete: <http://sourceforge.net/projects/prefuse/>

%\end{thebibliography}

%---koniec Referencii

% -------------------
%--- Prilohy---
% -------------------

\input attachments.tex

%\addcontentsline{toc}{chapter}{Appendix A}
%\input AppendixA.tex
%
%\addcontentsline{toc}{chapter}{Appendix B}
%\input AppendixB.tex

\end{document}






