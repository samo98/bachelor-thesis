\chapter{Niektoré rýchlo merateľné siete}

Vzhľadom na to, že sme si v minulej kapitole dokázali, že vo všeobecnosti
je naše metriky ťažké vypočítať, v tejto kapitole sa skúsime pozrieť na niektoré
konkrétne typy sietí. Pozrieme sa na siete za predpokladu, že by si uzly volili
kvórové rezy určitým spôsobom a skúsime sa zamyslieť ako veľmi by boli takéto
siete bezpečné.
Po analýze niektorých typov dostaneme lepší náhľad do tvorenia kvór a
kvórových rezov a budeme vedieť popísať niektoré osvedčené postupy na tvorenie
kvórových rezov aj pre všeobecné siete, tak aby bola jej odolnosť vysoká
aj keď ju nebudeme vedieť exaktne zrátať.

\section {Reťazové siete}

Najprv si ešte upravíme a zjednodušíme definíciu sietí, ktorými sa budeme zaoberať.

\paragraph{Zjednodušené NQK-siete} sú \textit{NQK-siete}, pre ktoré platí
$\forall v:\:Q_v=Q\land K_v=K$.

\vspace{3mm}
\textit{Reťazové siete} budú zjednodušené NQK-siete s nasledujúcou voľbou
vlastných kvórových rezov:

$$Q(v_i)= \begin{cases}
            [\{v_0, ..., v_Q\} \setminus \{v_i\}]^K & i < Q \\
            [\{v_{i-Q}, ..., v_{i-1}\}]^K & inak
          \end{cases}$$

Pričom uzly sme si očíslovali: $v_0, ..., v_{N-1}$.
Teda vlastné kvórové rezy uzlov sú ľubovoľné K-prvkové podmnožiny Q uzlov pred ním
(v zoradení podľa indexov) až na uzly, ktoré nemajú Q uzlov pred sebou, tie
použijú prvých Q uzlov v zoradení (okrem seba, aby to bol vlastný kvórový rez).

Všimnime si, že tým, že všetky uzly \uv{závisia} na uzloch pred nimi, tak
každé kvórum v sieti bude musieť obsahovať uzly s nízkym indexom.
Formálnejšie:
každý vrchol v kvóre musí mať v kvóre aj jeden zo svojich kvórových rezov
a teda ak v každom svojom kvórovom reze má uzol s menším indexom, tak
tento vrchol nemôže byť v žiadom kvóre uzol s najmenším indexom.

V našej sieti, ale jediné uzly s kvórovým rezom neobsahujúcim menšie indexy
sú uzly $v_0, ..., v_{Q}$, keď sa lepšie zamyslíme dokonca to budú len uzly
$v_0, ..., v_{Q-K}$, keďže každý kvórový rez obsahuje $K+1$ uzlov.
Teda ak si označíme uzol s najmenším indexom v kvóre ako $v_i$, tak platí
$i\leq Q-K$.

Všimnime si ale, že všetky tieto uzly majú kvórové rezy v množine $[\{v_0, ..., v_Q\}]^{K+1}$.
Čo teda znamená, že každé kvórum reťazovej siete obsahuje ako podmnožinu
niektorú $(K+1)$-prvkovú podmnožinu $\{v_0, ..., v_Q\}$.
Naopak zoberme si jednu $(K+1)$-prvkovú podmnožina $\{v_0, ..., v_Q\}$ a označme ju $A$.
Všetky prvky $A$ majú z definície kvórové rezy presne tieto $(K+1)$-prvkové podmnožiny,
a teda jeden kvórový rez pre každý prvok je samotná množina $A$. Teda dokonca $A$
je kvórum.

Ukázali sme teda, že všetky minimálne kvóra v reťazovej siete sú $[\{v_0, ..., v_Q\}]^{K+1}$.
Ako sme už dokázali v kapitole \ref{kap:security} bezpečnostný koeficient je
veľkosť ich najmenšieho prieniku.

Máme teda $(K+1)$-prvkové podmnožiny $(Q+1)$-prvkovej množiny a hľadáme ich
najmenší prienik z čoho dostaneme, že bezpečnostný koeficient
reťazovej siete je\\ $max (0,2K-Q+1)$.
Keďže reťazová sieť je špeciálnym typom NQK-siete, tak vieme aj
koeficient životaschopnosti a to je $Q-K+1$.

Teraz vieme povedať sieti ako zvoliť $Q$ a $K$ aby jej koeficient odolnosti bol
čo najväčší, teda hľadáme $\max\limits_{\forall K,Q} \: \min (2K-Q+1, Q-K+1)$.
\footnote{bezpečnostný koeficient rovný 0 sme nemuseli zbytočne uvažovať,
keďže vieme zvoliť $Q,K$ aby $2K-Q+1$ bolo kladné}

Táto funkcia nadobúda maximum pre $K = \frac{2}{3}Q$. A teda najodolnejšie
reťazové siete majú koeficient odolnosti rovný $\frac{Q}{3}+1$.
Teda optimum je zvoliť počet uzlov vo vlastných kvórových rezov rovný dvoch
tretín počtu uzlom ktorým vrchol dôveruje.

\section {Cyklické siete}

Ako bežne, uzly si očíslujme $v_0, ..., v_{N-1}$.
Keďže siete budú cyklické, často budeme používať pojem po sebe idúce uzly,
a takými budeme uvažovať aj uzly $v_0$ a $v_{N-1}$.
Ďalej si zadefinujeme aj intervaly uzlov, ktoré budeme označovať ako

$$<v_a, v_b> = \begin{cases}
                 \{v_a, ..., v_b\} & a<=b \\
                 \{v_a, ...,v_{N-1}, v_0,... v_b\} & inak
               \end{cases}
$$

Podobne označíme aj (polo)otvorené intervaly, kde len nebudeme uvažovať
hraničné uzly (tieto môžu byť potencionálne aj prázdne).

\textit{Cyklické siete} budeme označovať \textit{zjednodušené NQK-siete},
ktorých vlastné kvórové rezy budú definované nasledovne:

$$Q(v_i)= [<v_{(i+1) mod N}, v_{(i+Q) mod N}>]^K$$

Vlastné kvórové rezy sú teda ľubovoľné K-prvkové podmnožiny nasledujúcich $Q$ uzlov
za daným uzlom (v zoradení podľa indexov),
pričom indexy uzlov rátame cyklicky, teda modulo počet uzlov v sieti.

Tieto siete sú veľmi podobné už spomínaným \textit{reťazovým} sieťam, avšak
v nich sme využili, že všetky uzly \uv{záviseli} len na malej množine uzloch
a tam našli tie podstatné ktoré boli v každom kvóre.
V tomto type sietí je väčšia decentralizácia, keďže aj uzly s veľkým indexom
závisia zas na uzloch s malým indexom, takže tu budeme musieť analýzu siete
urobiť trikovejšie.

\paragraph{Veta}
Uvažujme $Q$ po sebe idúcich uzlov v cyklickej sieti $A=<v_i, v_{(i+Q) mod N})$.
Potom v každom jej kvóre je aspoň $K$ uzlov z $A$.

\paragraph{Dôkaz}
Zoberme si ľubovoľné kvórum $Q_1$ a pre spor predpokladajme, že menej ako
$K$ uzlov z tohto kvóra je v množine $A$.
Potom zrejme $v_{(i-1) mod N}\notin Q_1$, keďže všetky jeho vlastné kvórové rezy
sú K-prvkovou podmnožinou $A$, čo by bol spor.
Keďže každý kvórový rez má aspoň $K+1$ prvkov, tak existuje uzol $v_j$ taký že

$$v_j\in Q_1\land v_j\notin A\land \forall u\in\: <v_{(j+1) mod N}, v_i)\: :u\notin Q_1$$

Teda \uv{posledný} uzol z kvóra, ktorý je \uv{pred} množinou $A$.
V $Q_1$ musí ale byť aj niektorý vlastný kvórový rez $v_j$, teda $K$ vrcholov
z $<v_{(j+1)mod N}, v_{(j+Q) mod N}>$.\\
Rozlíšime 2 prípady:

\begin{enumerate}
  \item $v_i\in\: <v_{(j+1) mod N}, v_{(j+Q) mod N}>$

        V takomto prípade platí

        $<v_{(j+1) mod N}, v_{(j+Q) mod N}>\:=\: <v_{(j+1) mod N}, v_i)\: \cup <v_i, v_{(j+Q) mod N}>$.

        Pričom z definície $v_j$ žiadny uzol z $<v_{(j+1) mod N}, v_i)$ nie je v $Q_1$.
        Teda všetkých $K$ uzlov musí byť z $<v_i, v_{(j+Q) mod N}>\: \subseteq\: <v_i, v_{(i+Q) mod N})$,
        čo je presne čo sme chceli dokázať.
  \item $v_i\notin <v_{(j+1) mod N}, v_{(j+Q) mod N}>$

        V takomto prípade $Q_1$ neobsahuje žiadny kvórový rez $v_j$, keďže\\
        $<v_{(j+1) mod N}, v_{(j+Q) mod N}> \subseteq\: <v_{(j+1) mod N}, v_i$.
        Čo je spor s tým, že $Q_1$ obsahuje kvórový rez $v_j$ a teda je kvórum.
\end{enumerate} $\blacksquare$

\textbf{TODO}