\chapter{Metriky odolnosti siete}

V tejto kapitole si predstavíme našu metriku podľa ktorej budeme vyhodnocovať,
ktorá sieť je bezpečnejšia a životaschopnejšia na základe toho ako si uzly
zvolili kvórové rezy.
Ukážeme však, že spočítať tieto metriky na všeobecných sieťach sú ťažké
problémy a spočítať ich rýchlo na veľkých sieťach je nemožné.
Preto si predstavíme aj jeden špecifickejší typ siete, ktorý budeme v ďalších
kapitolách používať.

\paragraph {Bezpečnostný koeficient} Stellar systém $<\textbf{V},\textbf{Q}>$
má \textit{bezpečnostný koeficient} rovný $k$,
kde $k$ je najmenšie nezáporné číslo také, že existuje $k$ prvková množina
uzlov $B \in \textbf{V}$ pre ktorú platí, že $<\textbf{V}, \textbf{Q}>^B$ nie je
\textit{bezpečný}.\\
Ak také $k$ neexistuje (systém má prienik kvór nech odoberieme
ľubovoľnú množinu uzlov), tak si tento koeficient dodefinuje ako $|\textbf{V}|$.

\vspace{4mm}
Inými slovami: v sieti musí zlyhať aspoň $k$ uzlov aby sieť prestala byť
\textit{bezpečná}. Teda existuje $k$ prvková množina uzlov, ktorá nie je
\textit{nepotrebná}, konkrétne porušuje prvú podmienku \textit{nepotrebnej množiny}.
Naopak sieť dokáže prežiť aj napriek zlyhaniu $k-1$ ľubovoľných uzlov a ani
jeden uzol dodržujúci protokol nebude divergovať.
Samozrejme máme tu špeciálny prípad keď $k=0$, vtedy už samotná sieť nie je
\textit{bezpečná}.
V prípadoch keď je sieť dokonale bezpečná, teda odolná voči zlyhaniu ľubovoľnej
množiny uzlov si dodefinujeme koeficient na počet uzlov v systéme, keďže ak
už zlyhajú všetky uzly nemá žiadny ďalší zmysel merať bezpečnosť siete.

Podobne si vieme definovať koeficient pre životaschopnosť.

\paragraph {Koeficient životaschopnosti} Stellar systém $<\textbf{V},\textbf{Q}>$,
má \textit{koeficient životaschopnosti} rovný $k$, kde $k$ je najmenšie prirodzené
číslo také, že existuje $k$ prvková množina uzlov $B \in \textbf{V}$ pre ktorú
platí, že množina uzlov $\textbf{V} \setminus B$ v tomto Stellar systéme nie je
\textit{životaschopná}.

\vspace{4mm}
V sieti musí zlyhať aspoň $k$ uzlov aby sieť prestala byť
\textit{životaschopná}. Teda existuje $k$ prvková množina uzlov, ktorá nie je
\textit{nepotrebná}, konkrétne porušuje druhú podmienku \textit{nepotrebnej množiny}.
Naopak sieť dokáže prežiť aj napriek zlyhaniu $k-1$ ľubovoľných uzlov a všetky uzly
sa budú schopné ďalej podieľať na prijímaní konsenzu.
Špeciálny prípad $k=0$ zrejme nemôže nastať, keďže
$\textbf{V} \setminus \emptyset = \textbf{V}$ je kvórum v $<\textbf{V},\textbf{Q}>$.

Na záver si zadefinujeme koeficient, ktorý nám bude zaručovať obe podmienky spomenuté
vyššie. To čo si totiž môžeme uvedomiť je, že pokiaľ náš \textit{bezpečnostný koeficient}
je $k$, tak síce žiadne uzly dodržujúce protokol sme \uv{nestratili} v zmysle, že
neschvaľujú protichodné bloky aj keď teda $k-1$ uzlov zlyhalo. Napriek tomu táto
množina $k-1$ uzlov nemusí byť nepotrebná, keďže môže porušovať druhú podmienku a teda
niektoré uzly mohli prestať byť schopné podieľať sa na schvaľovaní blokov.
Podobne to funguje pri \textit{koeficiente životaschopnosti}.

\paragraph {Koeficient odolnosti} Stellar systém $<\textbf{V},\textbf{Q}>$,
má koeficient odolnosti rovný minimu z jeho \textit{bezpečnostného koeficientu}
a \textit{koeficientu životaschopnosti}.

\vspace{4mm}
Všimnime si, že pokiaľ sieť obsahuje menej zlyhaných uzlov ako je jej
\textit{koeficient odolnosti}, tak to nijak neovplyňuje chod samotnej siete a ani
jedného uzla dodržujúceho protokol. Keďže každá takáto množina je \textit{nepotrebná}.
Po zlyhaní ďalšieho uzla už ale žiadne garancie nie sú, systém môže byť narušený
aj nemusí, keďže tiež záleží, ktoré konkrétne uzly zlyhajú.

\section {Životaschopnosť}

Skúsme sa najprv zamerať na vyhodnocovanie metriky životaschopnosti siete.
V tomto prípade hľadáme najmenšiu (počtom prvkov) množinu uzlov $B$ takú, že
systém už nebude životaschopný aj \textit{napriek množine B}.
Teda, že bez uzlov z $B$ už ostatné uzly nebudú tvoriť kvórum v pôvodnom systéme.
Z definície kvóra vieme ale túto úlohu preformulovať. Kvórum musí obsahovať
aspoň jeden kvórový rez každého jeho prvku.
Je to teda ekvivalentné nájdeniu takého $B$, že niektorý uzol dodržujúci protokol
bude mať v každom \textit{vlastnom kvórovom reze} aspoň jeden prvok $B$.
Vtedy budeme hovoriť, že tento uzol prestal byť \textit{životaschopný}.
Nájsť takéto $B$ môžme ale pre každý vrchol zvlášť a potom pre sieť stačí vybrať
minimálne $B$. Veľkosť takéhoto minimálneho $B$ bude potom náš \textit{koeficient
životaschopnosti}.

Chceme teda nájsť takú najmenšiu množinu uzlov $B$, že pre uzol $u$ a pre každý
jeho \textit{vlastný kvórový rez} platí, že aspoň jeden jeho prvok je z $B$.
Toto je však známy problém pokrytia množín (angl. \textit{set cover problem})
\cite{karp1972reducibilitysetcover},
kde prvky sú všetky uzly nachádzajúce sa v ľubovoľnom vlastnom kvórovom reze $u$.
Množiny ktoré chceme pokryť sú vlastné kvórové rezy
a teda hľadáme najmenšiu množinu, ktorá tieto množiny pokryje.
O tomto probléme je známe, že je NP-úplný, čo nám teda hovorí, že tento koeficient
vo všeobecnosti nebudeme vedieť zistiť v polynomiálnom čase (ak $P\neq NP$), aj keď
na tento problém existujú rôzne rýchle a rôzne dobré aproximačné algoritmy,
ktoré by nám pomohli aspoň približne určiť koľko zlyhaným uzlov naša sieť prežije.

Ďalšou veľkou nevýhodou úplne ľubovoľných kvórových rezov je,
že už len popísanie týchto kvórových rezov môže byť exponenciálne veľké (od počtu uzlov),
čo nám výrazne ovplyvňuje výpočtovú silu. Preto mierne obmedzíme výber kvórových rezov
aby ich popis dokázal byť dostatočne krátky a zároveň dosť voľný (nenútil uzlov veriť
uzlom, ktorým nechcú). Na výmenu budeme o týchto sieťach vedieť lepšie a rýchlejšie
uvažovať.

Poďme sa teda radšej pozrieť na siete s konkrétnejšími kvórovými rezmi, pri ktorých
to budeme vedieť určiť.

\paragraph{NQK-siete} budeme nazývať siete s $N$ uzlami, kde každý uzol $u$ si
zvolí kvórové rezy tak, že si zvolí $Q_u$ uzlov ktorým verí a jeho vlastné kvórové rezy
budú všetky $K_u$ prvkové podmnožiny týchto $Q_u$ uzlov\footnote{Kvórové rezy
vyrobíme z vlastných kvórových rezov tak, že ku každému z nich pridáme uzol, ktorému
tieto kvórové rezy patria}.
Samozrejme máme prirodzené predpoklady na sieť, teda $Q_u < N$ a $1\leq K_u\leq Q_u$

\vspace{4mm}
V takýchto sieťach je ale jednoduché zistiť \textit{koeficient životaschopnosti}.

\paragraph{Veta}
Koeficient životaschopnosti \textit{NQK-siete} s množinou uzlov $V$ je
$\min\limits_{u\in V}(Q_u-K_u+1)$

\paragraph{Dôkaz}Z úvodu podkapitoly vieme, že stačí nájsť pre každý uzol
najmenšiu množinu, ktorá má aspoň jeden prvok v každom
\textit{vlastnom kvórovom reze} tohto uzla.
Zoberme si teda uzol $u$. A označme si množinu uzlov nachádzajúcich sa v aspoň
jednom jeho \textit{vlastnom kvórovom reze} ako $Q$ ($|Q|=Q_u$).
Pokiaľ zrejme zlyhá ľubovoľných $Q_u-K_u$ uzlov z $Q$, tak stále máme aspoň
jeden nepokrytý \textit{vlastný kvórový rez}, keďže v \textit{NQK-sieti}
každá $K_u$ prvková podmnožina $Q$ je \textit{vlastný kvórový rez}.

Naopak ak zlyhá $Q_u-K_u+1$ uzlov už neostane žiadna $K$ prvková podmnožina $Q$
(vlastný kvórový rez), ktorá by neobsahovala zlyhaný uzol.
Podľa úvodu podkapitoly teda máme, že koeficient životaschopnosti siete bude
minimum zo všetkých týchto výsledkov pre všetky uzly $u$. $\blacksquare$

\section {Bezpečnosť}

\label{kap:security}

Tentokrát sa pozrieme na metriku bezpečnosti siete. V tomto prípade sa snažíme
nájsť najmenšiu množinu, ktorá poruší prienik kvór v systéme. Čo sa dá preformulovať
na problém hľadania najmenšieho prieniku dvoch ľubovoľných kvór. Veľkosť tohoto
prieniku bude náš \textit{bezpečnostný koeficient} (pokiaľ v sieti existujú aspoň 2 kvóra,
ináč nastane špeciálny prípad z definície a koeficient bude počet uzlov v sieti).
Totiž stačí uvažovať práve množinu uzlov obsahujúcu práve tento prienik a tieto
dve kvóra po odstránení tejto množiny už prienik mať nebudú a teda celá sieť o túto
vlastnosť príde.
Keďže je to ale najmenší prienik, tak uvažovaním množiny menšieho počtu prvkov
nikdy neobsiahne celý prienik a teda aj po ich odstránení všetky prieniky ostanú.

Dokonca platí trošku príjemnejšie pravidlo, ak si zadefinujeme nasledovný pojem.

\paragraph {Minimálne kvórum} je také kvórum, že žiadna jeho vlastná podmnožina
už nie je kvórum.

\vspace{3mm}
\textit{Bezpečnostný koeficient} je dokonca veľkosť najmenšieho prieniku dvoch
ľubovoľných minimálnych kvór, keďže ak máme kvóra $Q_3\subseteq Q_1$ a $Q_4\subseteq Q_2$,
tak zjavne $|Q_3\cap Q_4|\leq |Q_1\cap Q_2|$.

\vspace{4mm}
Najprv si ukážeme, že pokiaľ nedáme žiadne obmedzenia na výber kvórových rezov,
tak zistenie nášho bezpečnostného koeficientu bude NP-ťažký problém, ba čo viac
dokonca len zistenie či náš koeficient je aspoň 1 bude NP-ťažké.
Dokážeme to redukciou nášho problému na známy NP problém, ktorým bude SAT
\cite{cook1971complexity}.

SAT je rozhodovací problém. Na vstupe má booleovskú formulu v konjuktívnej normálnej
forme a úlohou algoritmu je rozhodnúť, či existuje ohodnotenie premenných vo formule,
pri ktorom je vstupná formula splnená.

\paragraph{Veta}
Ak by sme vedeli zistiť v polynomiálnom čase pre daný Stellar systém, či má bezpečnostný
koeficient 0 (či existujú kvóra bez prieniku), tak by sme vedeli
v polynomiálnom čase zistiť či má SAT riešenie.

\paragraph{Dôkaz}
Majme teda na vstupe booleovskú formulu $X_1 \land X_2 ... \land X_m$, kde každé\\
$X_i = z_{i,1} \lor z_{i,2} ... \lor z_{i,p_i}$, kde
$z_{i,j} \in \{x_1, ..., x_n, \neg x_1, ..., \neg x_n\}$, teda máme konjuktívnu
normálnu formu z premenných $x_i$ a $\neg x_i$.

Teraz zostrojíme Stellar systém $<\textbf{V},\textbf{Q}>$, taký, že táto formula
je splniteľná práve vtedy keď zostrojený systém bude mať bezpečnostný koeficient 0.
Množina uzlov bude
$V=\{v_1, ...,v_{2n+1}, v_{1,1}, ..., v_{2n+1,m}, w_{1,1}, ..., w_{2n+1,n}\}$.

Aby sme nemuseli rozlišovať prípady premenných v pozitívnom a negatívnom
kontexte, tak $v_1, ..., v_n$ budú zodpovedať premenným v pozitívnom kontexte
($x_i$) a uzly $v_{n+1}, ..., v_{2n}$ premenným v negatívnom kontexte
($\neg x_{i-n}$) a napokon tu budeme mať špeciálny uzol $v_{2n+1}$.

Potom budeme mať uzly $v_{i,j}$, ktoré budú symbolizovať klauzulu $X_j$
a pre každý uzol ($v_k$) budeme mať kópiu (teda dokopy $2n+1$ kópií). Prečo potrebujeme
pre každý uzol ($v_k$) vlastnú kópiu uzla pre každú klauzulu, uvidíme pri zostrojovaní
kvórovej funkcie.
Samotnú redukciu by sme dokázali spraviť aj bez nich, no potrebovali by sme
príliš veľa kvórových rezov a redukcia by už nebola polynomiálna.

Napokon nám ostali špeciálne uzly $w_{i,j}$, ktorými sa budeme snažiť zabezpečiť
aby splniteľnosť formuly naozaj zodpovedala našej sieti a budeme ho používať
na to aby sme zabezpečili nejaký vzťah medzi $x_j$ a $\neg x_j$ a zas budeme mať
pre každý uzol reprezentujúci pôvodnú premennú vlastnú kópiu.

Kvórovú funkciu vracajúcu \textit{vlastné kvórové rezy} zostrojíme nasledovne:

\vspace{5mm}
$\hfill Q(v_i)=\{\{v_{i,1}, ..., v_{i,m}, w_{i,1}, ..., w_{i,n}\}\}\hfill$

$\hfill Q(v_{i,j})=\{\{v_k\} | x_k\in X_j\}\cup \{\{v_{k+n}\} | \neg x_k\in X_j\}\cup \{\{v_{2n+1}\}\}\hfill$

$\hfill Q(w_{i,j})=\{\{v_j\}, \{v_{j+n}\}\} \hfill$

\vspace{5mm}
To znamená, že všetky uzly $v_i$ majú jediný vlastný kvórový rez a to množinu
všetkých svojich zodpovedajúcich kópií všetkých klauzúl (plus špeciálne uzly
$w_{i,j}$).
Naopak všetky uzly zodpovedajúce niektorej inštancii klauzuly majú 1-prvkové
vlastné kvórové rezy. V nich uzly, ktoré zodpovedajú premenným
(výskyt premennej v klauzule sme zapísali symbolicky množinovým zápisom),
ktoré sa v klauzule nachádzajú (a špeciálny vrchol $v_{2n+1}$).
Na záver špeciálne uzly $w_{i,j}$ majú len 2 vlastné kvórové rezy a to uzly
reprezentujúce premennú $x_j$ a jej negáciu (zase ako pre klauzuly máme $2n+1$ kópií
pre každú premennú).

Sieť už máme skonštruovanú a zjavne táto redukcia sa dá spraviť polynomiálne,
keďže uzlov máme $(2n+1)(m+n+1)$ a každý uzol má najviac $n+1$ kvórových rezov
(presnejšie počet premenných v klauzule +1) a každý kvórový rez má veľkosť
najviac $m$. Teraz ostáva ukázať, že naozaj v tejto sieti existujú 2 kvóra bez
prienikov práve vtedy keď je booleovská formula splniteľná.

Najprv ukážeme, že ak booleovská formula je splniteľná, tak existujú 2 kvóra
bez prieniku.
Uvažujme jedno vhodné ohodnotenie premenných. Vytvorme si množinu\\
$A=\{i | x_i=1\}\cup \{n+i | x_i=0\}$. $A$ je vlastne množina indexov uzlov,
ktoré reprezentujú kladne ohodnotené premenné v našom ohodnotení.
Označme $A'$ ako doplnok $A$ do množiny všetkých platných indexov, teda
$A'=\{1,2,..., 2n+1\} \setminus A$.

Potom
$$\mathbf{Q_1}=\{v_a | a\in A\}\cup \{v_{a,j} | a\in A\}\cup \{w_{a,j} | a\in A\}$$
je kvórum.

Zrejme $v_a$ majú svoj jediný kvórový rez v $Q_1$ ($v_{a,j}, w_{a,j}$ pre všetky $j$).
Podobne aj $w_{a,j}$ tu má kvórový rez, keďže $Q_1$ obsahuje práve jeden
prvok z dvojice $v_i$ a $v_{n+i}$ (buď je $x_i$ ohodnotené 0 alebo 1).
A napokon aj kvórový rez $v_{a,j}$ je v $Q_1$, keďže
ak je formula splnená aspoň jeden literál klauzuly $X_j$ musí byť pravdivý a
naša množina obsahuje všetky uzly zodpovedajúce pravdivým literálom (ak bolo
$x_i$ pravdivé obsahujeme $v_i$, ináč obsahujeme $v_{n+i}$).
Teda $Q_1$ je naozaj kvórum.

Ukážeme, že aj
$$\mathbf{Q_2}=\{v_a | a\in A'\}\cup \{v_{a,j} | a\in A'\}\cup \{w_{a,j} | a\in A'\}$$
je kvórum.

Podobne ako v $Q_1$, množina obsahuje aspoň jeden kvórový rez uzlov $v_a$ a $w_{a,j}$.
\\
Keďže $v_{2n+1}\in Q_2$ a takisto aj $v_{2n+1}\in Q(v_{a,j})$, tak aj $Q_2$ je kvórum.
Tieto kvóra ale nemajú prienik, keďže $A$ a $A'$ nemajú.

Ostáva teda ukázať, že ak v našom systéme existujú 2 kvóra $Q_1$ a $Q_2$ bez prieniku,
tak naša formula je splniteľná.

Poskupinkujme všetky uzly podľa prvého indexu, tieto skupinky označme

$$\mathbf{V_i}=\{v_i, v_{i,1}, ..., v_{i,m}, w_{i,1}, ..., w_{i,n}\}$$.

Všimnime si, že ak je v nejakom kvóre $v_i$, tak potom aj celé $V_i$
($v_i$ má jediný kvórový rez).

Naopak ak je v kvóre iný uzol tejto skupiny a $v_i$ v kvóre nie je,
tak aj po odobratí tohto uzlu z kvóra, sa kvórum zachová, keďže $v_i$ je jediný uzol,
ktorý má ostatné uzly v skupine vo svojom kvórovom reze.

Preto pre jednoduchosť uvažujme kvóra $Q_1$ a $Q_2$, ktoré obsahujú iba celé skupiny $V_i$.
Toto môžeme spraviť keďže ako sme ukázali vyššie, stačí skúmať prieniky minimálnych kvór.

Bez ujmy na všeobecnosti nech $Q_1$ je kvórum neobsahujúce špeciálnu skupinu
$V_{2n+1}$ (keďže sú bez prieniku jedno z nich túto vlastnosť má).

Ohodnoťme premenné vo formule nasledovne:

$$x_i= \begin{cases}
            1 & V_i \subseteq Q_1 \\
            0 & V_{i+n} \subseteq Q_1
       \end{cases}$$

\vspace{5mm}
Všimnime si, že každá premenná je definovaná, keďže každé kvórum obsahuje skupinu
$V_i$ alebo $V_{i+n}$.
Totiž, každá skupina ($V_j$) obsahuje uzol $w_{j,i}$, ktorej jediné
kvórové rezy sú $\{v_i\}$ a $\{v_{i+n}\}$ a teda potom už nutne kvórum obsahuje
celú skupinu $V_i$ alebo celú skupinu $V_{i+n}$.

Avšak $Q_1$ a $Q_2$ nemajú prienik a keďže obe obsahujú aspoň jednu zo skupín $V_i$
a $V_{i+n}$, tak každé z nich obsahuje dokonca práve jednu skupinu a teda je ohodnotenie
premenných aj dobre definované.

Na záver ukážeme, že formula je splnená.
Keďže $Q_1 \neq \emptyset$, tak obsahuje skupinu $V_i$ pre nejaké $i$
a teda aj uzly $v_{i,1}, ..., v_{i,m}$.
Tieto uzly ale majú také kvórové rezy, že naše kvórum musí obsahovať aj uzly
z každej klauzuly $X_1, ..., X_m$.
Teda naozaj ak sú v našom systéme 2 kvóra bez prieniku pôvodná formula je
splniteľná. $\blacksquare$

\vspace{5mm}
Počas písania našej práce bol do predtlače publikovaný článok\cite{lachowski2019},
ktorý dokázal rovnaký výsledok. Dokonca ho na záver jednoduchou úvahou zosilnil
na tvrdenie, že už len zistiť či bezpečnostný koeficient siete je 0
v sieťach s najviac 2 kvórovými rezmi a 2 prvkami v kvórovom reze je NP-ťažké.

Môžeme si všimnúť, že naša redukcia dokonca používala \textit{NQK-sieť} definovanú v predošlej
podkapitole. Preto nebudeme vedieť rýchlo zisťovať tento koeficient ani v týchto sieťach
aj keď koeficient životaschopnosti zistiť vieme.
Budeme sa teda snažiť ešte viac spresniť pravidlá na výber kvórových rezov,
aby sme tento koeficient vedeli presne vyrátať a tak vedeli aj niečo o danej sieti povedať.