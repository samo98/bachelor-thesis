\chapter{Metriky odolnosti siete}

V tejto kapitole si predstavíme našu metriku podľa ktorej budeme vyhodnocovať,
ktorá sieť je bezpečnejšia a životaschopnejšia na základe toho ako si uzly
zvolili kvórové rezy.
Budeme tiež chcieť sa pozrieť na niektoré možné varianty volenia si kvórových
rezov. Vyrátame a  porovnáme ich metriky a tiež odôvodníme veľkosti týchto metrík.
Na základe tohto dostaneme bližší vhľad do volenia si kvórových rezov, tak aby
bola sieť čo najviac odolná voči potencionálnym útočníkom alebo nepredvídateľným
chybám v uzloch.

\paragraph {Bezpečnostný koeficient} Stellar systém $<\textbf{V},\textbf{Q}>$
má \textit{bezpečnostný koeficient} rovný $k$,
kde $k$ je najmenšie prirodzené číslo také, že existuje $k$ prvková množina
uzlov $B \in V$ pre ktorú platí, že $<\textbf{V}, \textbf{Q}>^B$ nie je
\textit{bezpečný}.

\vspace{4mm}
Inými slovami: v sieti musí zlyhať aspoň $k$ uzlov aby sieť prestala byť
\textit{bezpečná}. Teda existuje $k$ prvková množina uzlov, ktorá nie je
\textit{nepotrebná}, konkrétne porušuje prvú podmienku \textit{nepotrebnej množiny}.
Naopak sieť dokáže prežiť aj napriek zlyhaniu $k-1$ ľubovoľných uzlov a ani
jeden uzol dodržujúci protokol nebude divergovať.
Samozrejme máme tu špeciálny prípad keď $k=0$, vtedy už samotná sieť nie je
\textit{bezpečná}.

Podobne si vieme definovať koeficient pre životaschopnosť.

\paragraph {Koeficient životaschopnosti} Stellar systém $<\textbf{V},\textbf{Q}>$,
má \textit{koeficient životaschopnosti} rovný $k$, kde $k$ je najmenšie prirodzené
číslo také, že existuje $k$ prvková množina uzlov $B \in V$ pre ktorú platí, že
$<\textbf{V}, \textbf{Q}>^B$ nie je \textit{životaschopná}.

\vspace{4mm}
V sieti musí zlyhať aspoň $k$ uzlov aby sieť prestala byť
\textit{životaschopná}. Teda existuje $k$ prvková množina uzlov, ktorá nie je
\textit{nepotrebná}, konkrétne porušuje druhú podmienku \textit{nepotrebnej množiny}.
Naopak sieť dokáže prežiť aj napriek zlyhaniu $k-1$ ľubovoľných uzlov a všetky uzly
budú schopné sa ďalej podieľať na prijímaní konsenzu.
Špeciálny prípad keď $k=0$, znamená, že už samotná sieť nie je \textit{životaschopná}.

Na záver si zadefinujeme koeficient, ktorý nám bude zaručovať obe podmienky spomenuté
vyššie. To čo si totiž môžeme uvedomiť je, že pokiaľ náš \textit{bezpečnostný koeficient}
je $k$, tak síce žiadne uzly dodržujúce protokol sme \uv{nestratili} v zmysle, že
neschvaľujú protichodné bloky aj keď teda $k$ uzlov zlyhalo. Napriek tomu táto
množina $k$ uzlov nemusí byť nepotrebná, keďže môže porušovať druhú podmienku a teda
niektoré uzly mohli prestať byť schopné podieľať sa na schvaľovaní blokov.
Podobne to funguje naopak pri \textit{koeficiente životaschopnosti}.

\paragraph {Koeficient odolnosti} Stellar systém $<\textbf{V},\textbf{Q}>$,
má koeficient odolnosti rovný minimu z jeho \textit{bezpečnostného koeficientu}
a \textit{koeficientu životaschopnosti}.

\vspace{4mm}
Všimnime si, že pokiaľ sieť obsahuje menej zlyhaných uzlov ako je jej
\textit{koeficient odolnosti}, tak to nijak neovplyňuje chod samotnej siete a ani
jedného uzla dodržujúceho protokol. Keďže každá takáto množina je \textit{nepotrebná}.
Po zlyhaní ďalšieho uzla už ale žiadne garancie nie sú, systém môže byť narušený
aj nemusí, keďže tiež záleží, ktoré konkrétne uzly zlyhajú.

\section {Životaschopnosť}

Skúsme sa najprv zamerať na vyhodnocovanie metriky životaschopnosti siete.
V tomto prípade hľadáme najmenšiu (počtom prvkov) množinu uzlov $B$, ktorými
pokazíme životaschopnosť siete, teda, že bez nich už ostatné uzly nebudú
tvoriť kvórum v pôvodnom systéme.
Z definície kvóra vieme ale túto úlohu preformulovať. Kvórum musí obsahovať
aspoň jeden kvórový rez každého jeho prvku.
Je to teda ekvivalentné nájdeniu takého $B$, že niektorý uzol dodržujúci protokol
bude mať v každom kvórovom reze aspoň jeden prvok $B$.
Vtedy budeme hovoriť, že tento uzol prestal byť \textit{životaschopný}.
Nájsť takéto $B$ môžme ale pre každý vrchol zvlášť a potom pre sieť stačí vybrať
minimálne $B$. Veľkosť takéhoto minimálneho $B$ bude potom náš \textit{koeficient
životaschopnosti}.

Chceme teda nájsť takú najmenšiu množinu uzlov $B$, že pre uzol $u$ a pre každý
jeho kvórový rez platí, že aspoň jeden jeho prvok je z $B$.
Toto je však známy problém pokrytia množín (angl. \textit{set cover problem}),
kde prvky sú všetky uzly nachádzajúce sa v ľubovoľnom kvórovom reze $u$.
Množiny ktoré chceme pokryť sú kvórové rezy a teda hľadáme najmenšiu množinu,
ktorá tieto množiny pokryje.
O tomto probléme je známe, že je NP-úplný, čo nám teda hovorí, že tento koeficient
vo všeobecnosti nebudeme vedieť rýchlo zistiť, aj keď na tento problém existujú
rôzne rýchle a rôzne dobré aproximačné algoritmy, ktoré by nám pomohli aspoň
približne určiť koľko zlyhaným uzlov naša sieť prežije.

Poďme sa teda radšej pozrieť na siete s konkrétnejšími kvórovými rezmi, pri ktorých
to budeme vedieť určiť.

Ďalšou veľkou nevýhodou úplne ľubovoľných kvórových rezov je,
že už len popísanie týchto kvórových rezov môže byť exponenciálne veľké (od počtu uzov),
čo nám výrazne ovplyvňuje výpočtovú silu. Preto mierne obmedzíme výber kvórových rezov
aby ich popis dokázal byť dostatočne krátky a zároveň dosť voľný (nenútil uzlov veriť
uzlom, ktorým nechcú). Na výmenu budeme o týchto sieťach vedieť lepšie a rýchlejšie
uvažovať.

\paragraph{NQK-siete} budeme nazývať siete s $N$ uzlami, kde každý uzol $u$ si
zvolí kvórové rezy tak, že si zvolí $Q_u$ uzlov ktorým verí a jeho kvórové rezy
budú všetky $K_u$ prvkové podmnožiny týchto $Q_u$ uzlov.

\vspace{4mm}
V takýchto sieťach je ale jednoduché zistiť \textit{koeficient životaschopnosti}.

\paragraph{Veta}
Koeficient životaschopnosti \textit{NQK-siete} s množinou uzlou $V$ je
$\min\limits_{u\in V}(Q_u-K_u+1)$

\paragraph{Dôkaz}Z úvodu podkapitoly vieme, že stačí nájsť pre každý uzol
najmenšiu množinu, ktorá má aspoň jeden prvok v každom kvórovom reze tohto uzla.
Zoberme si teda uzol $u$. A označme si množinu uzlov nachádzajúcich sa v aspoň
jednom jeho kvórovom reze ako $Q$ ($|Q|=Q_u$). Pokiaľ zrejme zlyhá ľubovoľných
$Q_u-K_u$ uzlov z $Q$, tak stále máme aspoň jeden nepokrytý kvórový rez, keďže v
\textit{NQK-sieti} každá $K$ prvková podmnožina $Q$ je kvórový rez.

Naopak ak zlyhá $Q_u-K_u+1$ uzlov už neostane žiadna $K$ prvková podmnožina $Q$
(kvórový rez), ktorá by neobsahovala zlyhaný uzol.
Podľa úvodu podkapitoly teda máme, že koeficient životaschopnosti siete bude
minimum zo všetkých týchto výsledkov pre všetky uzly $u$. $\blacksquare$


\section {Bezpečnosť}

Tu to bude o niečo náročnejšie
