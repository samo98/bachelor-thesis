\chapter{Stellar konsenzus protokol}

\label{kap:stellar}

V tejto časti si predstavíme protokol \cite{mazieres2015stellar}, ktorý bude
spĺňať všetky popísané požiadavky v kapitole vyššie a bude zaručené,
že všetky správne fungujúce uzly sa dohodnú na rovnakej hodnote.

Každý uzol si bude udržiavať svoju kópiu blokovej reťaze. Jednu transakciu (a
metadáta o nej) v blokovej reťazi budeme nazývať \textit{blok}.
Cieľom tohto konsenzus protokolu teda je aby sa celá sieť vždy dohodla
na obsahu nového bloku a všetky uzly si ju mohli zapísať do svojej
blokovej reťaze. 
Budeme tiež predpokladať, že každý správne fungujúci uzol nikdy nebude tvrdiť
dve protichodné tvrdenia.

\section {Kvórum a typy uzlov}

Keď sa uzol dozvie od dostatočného množstva iných uzlov informáciu, uverí jej a
bude predpokladať, že žiadny fungujúci uzol nikdy nebude tvrdiť nič protichodné.
Takúto množinu uzlov pre uzol \textit{v} budeme nazývať kvórový rez. Keďže
sa nám po čase môže stať, že niektoré uzly zlyhajú, dovolíme mať každému vrcholu
viac kvórových rezov.

\paragraph {Stellar systém} budeme nazývať
dvojicu $<\textbf{V},\textbf{Q}>$ kde $\textbf{V}$ bude zoznam uzlov a $\textbf
{Q}$ bude funkcia priradzujúca uzlom \textit{kvórové rezy}.
\newline
Presnejšie $\textbf{Q} : \textbf{V} \to 2^{2^V} \setminus \emptyset$, \:
$\forall v \in \textbf{V}, \forall q \in \textbf{Q}(v), v \in q$
\newline
Teda každý vrchol obsahuje sám seba v každom svojom kvórovom reze.

\paragraph {Kvórum} je množina uzlov $U \subseteq V$ v
Stellar systéme práve vtedy keď $U \neq \emptyset$ a $U$ obsahuje kvórový rez pre
každý jeho prvok, teda $\forall v \in \textbf{U} \: \exists q \in \textbf{Q}
(v)$,
že $q \subseteq U$

\begin{flushleft}
Kvórum je množina uzlov dostatočná na dosiahnutie konsenzu. Totiž táto množina obsahuje
pre každý svoj prvok kvórový rez, ktorý ho dokáže presvedčiť.
Na obrázku \ref{obr:kvorum} je znázornená jedna možnosť ako si uzly mohli navoliť
kvórové rezy.
Všimnime si, že napríklad množina $\{v_2, v_3, v_4\}$ je síce kvórový rez uzlu
$v_4$, ale kvórum netvorí, keďže neobsahuje jediný kvórový rez $v_2$.
Naopak množina uzlov $\{v_1, v_2, v_3\}$ je kvórový rez uzlov $v_1, v_2, v_3$ a
teda aj tvorí kvórum.

\begin{figure}
\centerline{\includegraphics[width=0.8\textwidth]{images/kvorum.pdf}}
\caption[Príklad siete]{Príklad volenia kvórových rezov od uzlov} \label{obr:kvorum}
\end{figure}

\end{flushleft}

Uzly budeme rozdeľovať do dvoch hlavných skupín -- na uzly držiacich sa protokolu a
nedržiacich sa protokolu. Medzi uzly nedodržiavajúce protokol zaraďujeme buď
Byzantíncov (napríklad nepriateľ sa zmocnil uzlu) alebo uzly, ktoré akýmkoľvek
spôsobom zlyhali.
Cieľom protokolu je aby uzly dodržujúce protokol dosiahli konsenzus. Toto vieme
popísať dvomi vlastnosťami.

\paragraph {Bezpečnosť} Množina uzlov v Stellar systéme je \textit{bezpečná}, ak žiadne
dva uzly z nich neschvália pre jeden konkrétny blok iné hodnoty. Teda nebudú mať
rôzne verzie transakčnej histórie.

\paragraph {Životaschopnosť} Množina uzlov v Stellar systéme je životaschopná,
ak vie schvaľovať nové bloky do svojej blokovej reťaze, bez potreby spolupráce
zlyhaných uzlov.

\vspace{4mm}
Teraz uzly dodržujúce protokol rozdelíme do troch skupín. Uzly, ktoré sú aj
bezpečné aj životaschopné budeme označovať ako \textit{korektné}.
Uzly, ktoré nie sú bezpečné budeme označovať ako \textit{divergentné} a uzly,
ktoré síce sú bezpečné ale nie sú životaschopné zase budeme označovať ako
\textit{zablokované}.
Všimnime si, že aj uzly dodržujúce protokol môžu byť divergentné, stačí, že
hlasovaním spolu s Byzantíncami sa im poškodila bloková reťaz. Práve takýmto
situáciám sa bude Stellar protokol snažiť vyhnúť.
Vizualizáciu katogorizácie uzlov môžeme vidieť na obrázku \ref{obr:typy_uzlov}.

\begin{figure}
\centerline{\includegraphics[width=0.8\textwidth]{images/rozdelenie_uzlov}}
\caption[Kategorizácia uzlov]{Rozdelenie uzlov do kategórie podľa schopnosti
zúčastňovať sa na konsenze} \label{obr:typy_uzlov}
\end{figure}

\section {Vlastnosti kvór a uzlov}

Keby si uzly určovali svoje kvóra úplne hocijako, sieť by vôbec nemusela byť
prepojená a teda by mohla byť divergentná. Preto budeme požadovať aby kvóra
spĺňali niektoré vlastnosti.

\paragraph {Prienik kvór} Stellar systém má prienik kvór práve vtedy ak každá
dvojica kvór má aspoň jeden spoločný uzol.

\begin{figure}
\centerline{\includegraphics[width=0.8\textwidth]{images/prienik_kvor.pdf}}
\caption[Príklad siete bez prieniku kvór]{Sieť bez prieniku kvór môže
divergovať} \label{obr:prienik_kvor} \end{figure}

\vspace{4mm}
Na obrázku \ref{obr:prienik_kvor} môžeme vidieť príklad siete bez prieniku kvór.
Disjunktné množiny $\{v_1, v_2\}$ a $\{v_3, v_4\}$ sú kvóra bez prieniku. Takže
tieto množiny sú samé o sebe schopné schvaľovať bloky a predlžovať blokovú
reťaz.
Keďže ale nemajú prienik nemusia schváliť rovanký blok a teda sieť bude
divergovať.

Teraz si definujme operáciu mazania množiny uzlov zo Stellar systému.

\paragraph {Operácia zmaž} Ak máme Stellar systém $<\textbf{V},\textbf{Q}>$ a
zmažeme z neho množinu uzlov $B \subseteq V$, dostaneme systém $<\textbf{V},
\textbf{Q}>^B \: = \: <\textbf{V} \setminus B,\textbf{Q}^B>$, kde $\textbf{Q}^B
= \{q \setminus B \: | \: q \in \textbf{Q}(v)\}$.

\vspace{4mm}
Takto si ľahko vieme zúžiť systém na množinu uzlov, ktoré chceme ďalej uvažovať
a pomocou toho analyzovať stabilnosť siete aj po prehlásení niektorých uzlov
za nedôveryhodné. Ako sme si už hovorili tak aby bolo kvórum schopné dosiahnuť
konsenzus a budovať rovnakú blokovú reťaz, musí mať prienik kvór. Avšak ak je
tento prienik práve byzantínsky uzol, tak naša sieť bude aj tak divergentná,
keďže práve on môže poslať jednému kvóru že hlasoval \textit{za} a druhému,
že hlasoval \textit{proti} a oba môžu schváliť protichodné tvrdenia.
Preto si dodefinujeme \textit{nepotrebné množiny}.
Budeme chcieť mať sieť rozloženú tak, aby všetky zlyhané uzly boli v
nepotrebnej množine a teda sieť bola schopná prijímať konsenzus aj bez nich.

\pagebreak

\paragraph {Nepotrebná množina} Ak máme Stellar systém $<\textbf{V},
\textbf{Q}>$, tak množina uzlov $B$ je nepotrebná práve vtedy keď
\begin{itemize}
\item  $<\textbf{V}, \textbf{Q}>^B$ má prienik kvór
\item  $\textbf{V} \setminus B$ je $\pmb{\emptyset}$ alebo kvórum v $<
\textbf{V}, \textbf{Q}>$
\end{itemize}

Prvá podmienka chráni sieť pred divergenciou, ktorú by mohli spôsobovať uzly z
$B$, tým že by schvaľovali protichodné tvrdenia. Tým, že prienik kvór ostane
zachovaný aj keď uzly z $B$ uvažovať nebudeme, sieť naozaj divergovať nebude.

Naopak druhá podmienka zachováva životaschopnosť siete bez ohľadu na uzly z $B$,
keďže zaručuje, že aj bez uzlov z $B$ vedia ostatné uzly schvaľovať nové bloky.

Čo si môžeme všimnúť je, že uzly musia balansovať vo veľkosti vyberaných kvórových
rezov. Keď si totiž uzly vyberú veľké kvórové rezy, tak vzniknú veľké kvóra.
Toto potom vedie k veľkým prienikom kvór (musí zlyhať veľa uzlov na
zničenie prieniku kvór), čo nám zaručuje malú šancu nepriateľa zničiť bezpečnosť
siete. Avšak oveľa ľahšie ohrozí jej životaschopnosť, keďže zlyhanie už len
málo uzlov ovplyvní veľa kvór. Podobne naopak pri malých kvórach vzniknú menšie
kvórové prieniky, čo zjednoduší nepriateľovi ohroziť bezpečnosť siete, avšak
bude musieť zlyhať viac uzlov aby to ohrozilo životaschopnosť siete
(aby zlyhali uzly v čo najviac kvórach).
Uzly v sieti teda musia nájsť balans pre čo najväčšiu obranyschopnosť siete.

\section {Priebeh schvaľovania bloku}

Schvaľovanie nového bloku prebieha vo viacerých fázach. Počas tohto procesu sa 
z pohľadu každého uzlu môže blok náchádzať v 3 rôznych stavoch -
\textit{neznámy}, \textit{akceptovaný}, \textit{potvrdený}.

Na začiatku schvaľovacieho procesu je pre každý uzol blok \textit{neznámy}.
Počas celého priebehu schvaľovania si uzly vymieňajú správy, na základe ktorých
si môže blok zvýšiť stav bloku (pre daný uzol môže ísť blok len do vyššieho
stavu, teda ak už pre neho bol blok \textit{akceptovaný}, nemôže ho blok
prehlásiť spätne za \textit{neznámy}).
Správy si môžeme predstavovať ako hlasovanie uzlov, či má alebo nemá byť
blok pridaný do blokovej reťaze.
Keď pre každý korektný uzol bude blok \textit{potrdený}, môžu si ho pridať
do svojej blokovej reťaze.

Pre potreby našej práce však nebudeme potrebovať vedieť ako a aké správy si medzi
sebou uzly budú vymieňať.
Uzly si budú meniť stavy bloku na základe správ od uzlov zo svojich kvórových rezov
spomínaných vyššie, avšak detajly presných pravidiel na základe ktorých sa tieto
stavy budú meniť využívať nebudeme. Plnohodnotne nám bude stačiť intuícia zavedenia
kvórových rezov a kvór.

Presnješie detajly Stellar konsenzus protokolu, procesu schvaľovania bloku,
samotné dôkazy, že takéto schvaľovanie vedie ku v praxi plne funkčnej sieti a
napokon aj samotnú implementáciu môže čitateľ nájsť v \cite{mazieres2015stellar}.
