\chapter*{Úvod} % chapter* je necislovana kapitola
\addcontentsline{toc}{chapter}{Úvod} % rucne pridanie do obsahu
\markboth{Úvod}{Úvod} % vyriesenie hlaviciek

V posledných rokoch nadobudli kryptomeny na popularite a výskum v tejto
oblasti sa raketovým tempom začal posúvať. Dnes už máme tisícky rôznych
kryptomien, z ktoých každá svojím spôsobom vylepšuje dnešnú existujúcu
finančnú sieť. Najväčším prínosom je schopnosť prevodu peňazí bez
centrálnej autority, no aj tak tu ostalo mnoho výziev, ktorými by sme
dokázali zlepšiť aktuálne podmienky v sieťach jednotlivých kryptomien.

My sme sa rozhodli bližšie pozrieť na kryptomenu Stellar. Pre túto
kryptomenu bol pred 3 rokmi vyvinutý nový konsenzus protokol. Tento
protokol mal za úlohu zachovať všetky dobré vlastnosti doteraz vyvinutých
protokolov, zrýchliť peňažné transakcie a najmä zabezpečiť otvorené
členstvo pre nové entity, ktoré sa chcú do siete pridať. Poslednú vlastnosť
dovtedy žiadny konsenzus protokol bez centrálnej autority zabezpečiť nevedel.

Jednou z dôležitých vlastností Stellar konsenzus protokolu je, že sa dokáže
vysporiadať aj so zlyhaním niekoľkých entít a udržať sieť bez poškodenia
schopnú naďalej vykonávať transakcie.
Koľko takýchto zlyhaní však vie sieť prežiť závisí od rôznych parametrov siete
a dodnes nie je známe ako rýchlo vieme pre konkrétnu sieť zistiť tento počet.

V našej práci si najprv predstavíme jednotlivé vlastnosti, ktoré od protokolu
budeme očakávať a následne si predstavíme samotný Stellar konsenzus protokol.
Potom si zadefinujeme jednotlivé metriky podľa ktorých budeme merať odolnosť
siete voči zlyhaniam a ukážeme, že zmerať tieto metriky vo všeobecnosti
nebudeme vedieť v polynomiálnom čase od veľkosti popisu siete.
Následne sa preto pozrieme na konkrétnejšie typy sietí kde tieto metriky
budeme vedieť určiť a dostaneme lepší pohľad na závislosť metrík od stavby siete.
Na záver rozanalyzujeme koľko zlyhaní dokážu zhruba prežiť náhodné siete.
A vyvinieme odporúčanie ako nastaviť niektoré parametre a pravidlá aby
sieť mohla očakávať, že v priemernom prípade bude schopná prežiť čo najväčší
počet zlyhaných entít.
