\chapter*{Záver}  % chapter* je necislovana kapitola
\addcontentsline{toc}{chapter}{Záver} % rucne pridanie do obsahu
\markboth{Záver}{Záver} % vyriesenie hlaviciek

V tejto práci sme si v úvode predstavili problematiku dosahovania konsenzu,
ukázali sme si ako využiť riešenie tohto problému vo finančnej sieti, kde
často musíme riešiť podobné problémy a rátať s nedôveryhodnými entitami v sieti.
Tiež sme si zhrnuli nedostatky existujúcej finančnej siete a takisto doteraz
známych protokolov, ktoré využívali na vykonávanie transakcií konsenzus
alebo aj iné metódy, ktoré sú medzi dnešnými kryptomenami populárnejšie.

Následne sme si predstavili samotný Stellar konsenzus protokol a porovnali sme
ho s už predstavenými inými metódami a ukázali si, ktoré problémy iných metód
rieši. Pri vylepšení samotného problému konsenzu išlo o zbavenie sa centrálnej
autority pri vstupe novej entity do siete.
Zoznámili sme sa so samotným fungovaním Stellar konsenzus protokolu a zistili
sme podmienky pri ktorých sieť dokáže odolávať útočníkom.

Sformulovali sme si presnejšie metriky na hodnotenie siete podľa toho koľko
zlyhaní uzlov ju nedokáže ovplyvniť. Dokázali sme si, že pre všeobecné siete
je zisťovanie týchto metrík NP-ťažký problém a teda sme si spresnili model
siete aby sme vedeli o sieti a jej odolnosti povedať aspoň niečo.

Vybrali sme si konkrétne modely \textit{reťazových} a \textit{cyklických} sietí,
zrátali ich metriky a dostali lepší vhľad do tvorenia kvór v takýchto sieťach.
Pri \textit{reťazových sieťach} sme zistili, že celá sieť je závislá na menšej
množine uzlov. Táto centralizácia však nebola až taká nepríjemná, keďže sieť
dokázala odolať aj zlyhaniu $\frac{1}{3}$ z týchto uzlov.
Pri \textit{cyklických sieťach} už sme na žiadnu centralizáciu nenarazili,
dokonca sme si ukázali, že sieť by bola v istých parametroch schopná prežiť
napadnutie až $\frac{1}{2}$ dôveryhodných uzlov, ktoré si konkrétny uzol vybral.
Táto sieť je však prudko symetrická a tesne zviazaná, čo sa nám v reálnej
sieti pravdepodobne nestane.

Na záver sme sa pozreli na to ako by dopadla odolnosť siete ak by boli siete
vytvorené náhodne. Najprv pri malých sieťach, kde sme metriky vedeli vypočítať
presne a potom aj na väčších kde sme však dokázali tieto metriky vypočítať
len približne.

Štatistiky z vybraných typov sietí a neskôr aj z náhodných sietí ukazujú, že
dokážeme získať odolnosť voči zlyhaniu až $\frac{1}{3}$ entít, čím sa dokážeme
porovnávať aj s centralizovanou verziou protokolov na konsenzus. Zistili sme,
že takúto vysokú odolnosť dosahujeme keď konkrétna entita je ochotná nechať sa
presvedčiť okolo $\frac{2}{3}$ entitami, ktorým dôveruje (voľba $K$ v našich
\textit{NQK-sieťach}). Naopak voľba $K < \frac{Q}{2}$ sa ukázala ako nevhodná,
keďže väčšina takto vygenerovaných sietí bola nepoužiteľná s bezpečnostným
koeficientom 0 a teda sieť nedávala žiadne garancie aj keď sa všetky uzly
správali podľa protokolu.

Napriek tomu sme dokázali analyzovať len siete s počtom entít do $50$, čo
nám dáva veľký priestor na zlepšenie. Takisto naša metóda pri väčších sieťach
nemusela byť veľmi presná a už len lepšie zrátať jej presnosť a tak zlepšiť
odhady našich metrík môže dať oveľa lepší pohľad do bezpečnosti a životaschopnosti
sietí založených na Stellar konsenzus protokole.